\section{Workflow}

\subsection{Erläuterung}
Wie bereits in Kapitel \ref*{sec:einfuehrung.grundbegriffe} erwähnt, ist ein dezentrales Versionskontrollsystem. Dadurch ist jeder Nutzer in der Lage ohne Abhängigkeit zu einem Server seinen Quellcode zu versionieren.

Jeder Nutzer kann sein eigenes Projekt durch \textit{git init} zu einem Git Repository initialisieren oder ein bestehendes Repository aus einem Online-Verzeichnis klonen. Beim klonen des Repositories wird der komplette Hierarchiebaum bzw. die Chronik mit jedem Commit und Branch geklont. Beim initialisieren eines neuen Repositories ist der Hierarchiebaum leer und unbeschrieben. Die Initialisierung eines neues Repositories wird bevorzugt beim neuen Projekten.

Um jedoch einen Workflow einzurichten, der es ermöglicht an bestehenden Projekten teil zu nehmen, muss zuerst das Projekt auf das lokale System geklont werden. Nachdem das Projekt als Git Repository auf dem lokalen System vorhanden ist, können Änderungen am Quellcode vorgenommen werden und in den Hierarchiebaum bzw. dem Repository hinzugefügt werden. Ist der Entwickler der Meinung, das alle Änderungen abgeschlossen sind, obliegt ihm selbst ob er seine Änderungen an den Hauptverantwortlichen einreicht. 

Der Entwickler kann seine Änderungen als Patch per E-Mail einreichen, sofern die Hauptverantwortlichen des geklonten Projekts dieses Verfahren unterstützen. Die bevorzugte Methode ist bei den meisten Online-Plattformen jedoch der Merge Request, manchmal auch Pull-Request genannt. 

Bei einem Merge-Request fragt man an, ob eine bestimmte Sammlung von Commits, die eine Änderung des Quellcodes wieder spiegeln, durch die Hauptverantwortlichen übernommen werden. Dadurch wird der Hierarchiebaum des Online-Repositories um die eigenen Commits erweitert. Anderen Entwickler, die auch dieses Repository geklont haben, können zukünftig die Änderung aus dem Online-Repository auf ihr eigenes lokales Repository übernehmen.  

\subsection{Einrichtung}
Im Heimatverzeichnis wird wie von den meisten IDE's bevorzugt, das Verzeichnis \textit{workspace} erstellt. Dort wird von einem Online-Repository dieses \LaTeX Projekt, das dieses Dokument enthält geklont. Nach dem klonen befindet sich im Verzeichnis \textit{workspace} der Unterordner \textit{sdsm\_ss17\_git}, in den mittels \textit{cd} nachträglich navigiert wird.

\begin{minted}[linenos, framesep=2mm, fontsize=\small]{bash}
$ mkdir ~/workspace
$ git clone https://git.cryptic.systems/fh-trier/sdsm_ss17_git.git
$ cd sdsm_ss17_git
\end{minted}

\begin{INFO}
  Zum klonen eines Online-Repositories wird auf vielen Online-Plattformen wie \href{https://github.com}{GitHub} oder \href{https://gitlab.com}{GitLab} eine Adresse per \textit{HTTPS-} oder \textit{SSH-}Protokoll angeboten. 
  
  Für eine Verbindung per SSH-Protokoll muss ein Account auf diesen Online-Plattformen registriert und der öffentliche Schlüssel des asymmetrischen Verschlüsselungsverfahren hinterlegt sein.  
\end{INFO}

Das heruntergeladene Repository spiegelt nun den Zustand ab, auf dem sich die Referenz \textit{HEAD} befindet. Um dies zu überprüfen, bietet Git den Befehl \textit{git log} an. Wir lassen uns die neusten zwei Commits anzeigen.

\begin{minted}[linenos, framesep=2mm, fontsize=\small]{bash}
$ git log -2
commit 4660915d6c031598c77b49c6275e426f7c3a85c3 (HEAD -> master, 
origin/master, origin/HEAD)
Author: Markus Pesch <markus.pesch@cryptic.systems>
Date:   Thu Mar 15 16:49:12 2018 +0100

add: chapter 1

commit 1ca84c7271f3f3c306b779256bd1902497f44fb9
Author: Markus Pesch <markus.pesch@cryptic.systems>
Date:   Thu Mar 15 13:26:02 2018 +0100

fix: geometry
\end{minted}

Zu erkennen ist, dass sich die Referenz \textit{HEAD} auf der gleichen Position wie der Branch \textit{master} befindet. Der Branch \textit{master} ist ein Branch auf dem lokalen System des Entwicklers. Die Referenz \textit{origin/master} referenziert auf den Commit, auf dem das Online-Repository \textit{origin} und dessen Branch \textit{master} steht. Alle Referenzen referenzieren auf den Commit mit der ID \textit{ea39aa0}.

Um zu überprüfen welche Adresse sich hinter dem Label \textit{origin} befindet, bietet Git den Befehl \textit{git remote -v} an.

\begin{minted}[linenos, framesep=2mm, fontsize=\small]{bash}
$ git remote -v
origin  git@git.cryptic.systems:fh-trier/sdsm_ss17_git.git (fetch)
origin  git@git.cryptic.systems:fh-trier/sdsm_ss17_git.git (push)
\end{minted}
 
Möchte man eine Remote-Verbindung entfernen kann dies unter Angabe des Git Befehls \textit{git remote remove  $ < $label $ > $} vorgenommen werden. 
  
Hinzufügen von neuen Remote-Verbindungen sind möglich mittels \textit{git remote add $ < $label$ > $ $ < $adresse$ > $}.

\begin{INFO}
  Möchte man den Quellcode eines existierenden Projektes verbessern und an die Verantwortlichen einreichen, muss der Workflow abgeändert werden.
  
  
\end{INFO}