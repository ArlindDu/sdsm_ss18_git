\section{Workflow einrichten}
Wir erstellen im Heimatverzeichnis, wie von den meisten IDE's bevorzugt, das Verzeichnis \textit{workspace} und klonen von einem Online-Repository dieses Projekt. Anschließend befindet sich im Verzeichnis \textit{workspace} der Unterordner \textit{sdsm\_ss17\_git} in den mittels \textit{cd} anschließend navigiert wird.

\begin{minted}[linenos, framesep=2mm, fontsize=\small]{bash}
$ mkdir ~/workspace
$ git clone https://git.cryptic.systems/fh-trier/sdsm_ss17_git.git
$ cd sdsm_ss17_git
\end{minted}

\begin{INFO}
  Zum klonen eines Online-Repositories wird auf vielen Online-Plattformen wie \href{https://github.com}{GitHub} oder \href{https://gitlab.com}{GitLab} eine Adresse per \textit{HTTPS-} oder \textit{SSH-}Protokoll angeboten. 
  
  Für eine Verbindung per SSH-Protokoll muss ein Account auf diesen Online-Plattformen registriert und der öffentliche Schlüssel des asymmetrischen Verschlüsselungsverfahren hinterlegt sein.  
\end{INFO}

Das heruntergeladene Repository spiegelt nun den Zustand ab, auf dem sich die Referenz \textit{HEAD} befindet. Um dies zu überprüfen, bietet Git den Befehl \textit{git log} an. Wir lassen uns die neusten zwei Commits anzeigen.

\begin{minted}[linenos, framesep=2mm, fontsize=\small]{bash}
$ git log -2
commit 4660915d6c031598c77b49c6275e426f7c3a85c3 (HEAD -> master, 
origin/master, origin/HEAD)
Author: Markus Pesch <markus.pesch@cryptic.systems>
Date:   Thu Mar 15 16:49:12 2018 +0100

add: chapter 1

commit 1ca84c7271f3f3c306b779256bd1902497f44fb9
Author: Markus Pesch <markus.pesch@cryptic.systems>
Date:   Thu Mar 15 13:26:02 2018 +0100

fix: geometry
\end{minted}

Zu erkennen ist, dass sich die Referenz \textit{HEAD} auf der gleichen Position wie der Branch \textit{master} befindet. Der Branch \textit{master} ist ein Branch auf dem lokalen System des Entwicklers. Die Referenz \textit{origin/master} referenziert auf den Commit, auf dem das Online-Repository \textit{origin} und dessen Branch \textit{master} steht. Alle Referenzen referenzieren auf den Commit mit der ID \textit{ea39aa0}.

Um zu überprüfen welche Adresse sich hinter dem Label \textit{origin} befindet, bietet Git den Befehl \textit{git remote -v} an.

\begin{minted}[linenos, framesep=2mm, fontsize=\small]{bash}
$ git remote -v
origin  git@git.cryptic.systems:fh-trier/sdsm_ss17_git.git (fetch)
origin  git@git.cryptic.systems:fh-trier/sdsm_ss17_git.git (push)
\end{minted}
 
Möchte man eine Remote-Verbindung entfernen kann dies unter Angabe des Git Befehls \textit{git remote remove  $ < $label $ > $} vorgenommen werden. 
  
Hinzufügen von neuen Remote-Verbindungen sind möglich mittels \textit{git remote add $ < $label$ > $ $ < $adresse$ > $}.

\begin{INFO}
  Möchte man den Quellcode eines existierenden Projektes verbessern und an die Verantwortlichen einreichen, muss der Workflow abgeändert werden.
  
  
\end{INFO}