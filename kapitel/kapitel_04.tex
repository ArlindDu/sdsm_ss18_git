\section{Arbeiten mit Git}

\subsection{Erste Schritte}
Nachdem wie in Abschnitt \ref{workflow.einrichtung} beschrieben die generelle Umgebung eingerichtet wurde, befindet sich im gleichen Verzeichnis die Datei \textit{test.sh}.

\begin{minted}[framesep=2mm, linenos, fontsize=\small]{bash}
#!/usr/bin/env bash
PWD="$(pwd)"
ls -la ${PWD} | grep -e '^d' 2>/dev/null
\end{minted}

Nun ändern wir den Quellcode ab in Zeile drei und ersetzen \textit{grep -e '\^{}d'} durch \textit{ grep -e '\^{}-'}. Nach speichern sollte beim Ausführen der Datei nur Dateien angezeigt werden. Die Modifikation der Datei kann Git auflösen mittels \textit{\nameref{git-commands.advanced.diff}}


