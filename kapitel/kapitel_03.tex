\label{git-commands}
\section{Git-Befehle}
Um am Workflow, wie er in Abbildung \ref{img:workflow} abgebildet ist teil zu nehmen, sind ein paar grundlegende Git-Befehle notwendig. Ein paar Befehle wurden bereits in Kapitel \ref{sec:einfuehrung.git} und \ref{workflow.einrichtung} vorgestellt.

\label{git-commands.advanced}
\subsection{Weitere Git-Befehle}

\label{git-commands.advanced.add}
\myparagraph{git add}

\label{git-commands.advanced.commit}
\myparagraph{git commit}

\label{git-commands.advanced.fetch}
\myparagraph{git fetch}

\label{git-commands.advanced.push}
\myparagraph{git push}

\label{git-commands.advanced.merge}
\myparagraph{git merge}

\label{git-commands.advanced.rebase}
\myparagraph{git rebase}

\label{git-commands.advanced.reset}
\myparagraph{git reset}

\label{git-commands.advanced.revert}
\myparagraph{git revert}

\label{git-commands.advanced.log}
\myparagraph{git log}

\label{git-commands.advanced.status}
\myparagraph{git status}

\label{git-commands.advanced.diff}
\myparagraph{git diff}