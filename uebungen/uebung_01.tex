\section{Übung}
\label{sec:uebung_01}

% ##########################################################################
% ############################### Aufgabe 01 ###############################
% ##########################################################################
\subsection{Aufgabe}
\label{sec:uebung_01.aufgabe_01}
Starten Sie das Skript.

\subsubsection*{Lösung}
\label{sec:uebung_01.aufgabe_01.loesung}

\subsubsection*{Minted}


\section{Minted}

\subsection{Begin-Block}

\subsubsection{SQL-Code}
\begin{sqlcode}
  SELECT *
  FROM tab;
\end{sqlcode}

\subsubsection{AWK-Code}
\begin{awkcode}
  if NR > 3 {
    print "0";
  }
\end{awkcode}


\subsection{Inputfile}

\subsubsection{SQL-Code}
\inputsql{sql/test.sql}

\subsection{Inline}
\subsubsection{SQL-Code}
\sqlinline{SELECT * FROM tab;}

\section{Info-Boxen}
\begin{warn-popup}
  Warn-Popup
\end{warn-popup}

\begin{info-popup}
  Info-Popup
\end{info-popup}

\begin{example-popup}
  Example-Popup
\end{example-popup}


\section{Akronyme}
\acrfull{acr:sql}

\section{Text-Boxen}
\begin{textblock*}{\textwidth}(4cm,0cm)
  Ich bin eine Textbox
\end{textblock*}

\section{Aufzählungen}
\begin{itemize}[itemsep=0pt]
  \item Item
  \item Item
  \item Item
\end{itemize}

\begin{enumerate}[itemsep=0pt]
  \item Item
  \item Item
  \item Item
\end{enumerate}

\begin{itemize}[itemsep=0pt]
  \item[A)] Item
  \item[B)] Item
  \item[C)] Item
\end{itemize}
